\documentclass{article}

\usepackage[portuguese]{babel}
\usepackage[utf8]{inputenc}

\title{Relatório BDAD}
\date{31-3-2015}
\author{João Cabral, up20130495\\
	   João Mota, up201303462\\
	   Luís Morais, up2013xxxxx}

\begin{document}

\maketitle
\pagenumbering{gobble}
\newpage
\pagenumbering{arabic}

\section{Esquema Relacional}

Prisioneiro(\underline{número}, nome $\rightarrow$ Pessoa, data de entrada, data de saída, cela $\rightarrow$ Cela)\\
Funcionário(\underline{IDFuncionário}, nome $\rightarrow$ Pessoa, cargo, hora de entrada, hora de saída)\\
Tempo de Recreio(\underline{motivo $\rightarrow$ Recompensa}, duração, prisioneiro $\rightarrow$ Prisioneiro)\\
Objeto Pessoal(\underline{motivo $\rightarrow$ Recompensa}, duração, prisioneiro $\rightarrow$ Prisioneiro)\\
Local(\underline{nome}, função)\\
Cela(\underline{número}, nome $\rightarrow$ local)\\
Pena(\underline{IDPena}, data de sentença, motivo, prisioneiro $\rightarrow$ Prisioneiro)\\
Incidente(\underline{IDIncidente}, descrição, data, local $\rightarrow$ Local, relator $\rightarrow$ Funcionário )\\
PrisioneiroIncidente(prisioneiro $\rightarrow$ Prisioneiro, incidente $\rightarrow$ Incidente)\\


%Cmo se faz pra classes d atribuição?


\end{document}