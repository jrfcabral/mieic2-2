\documentclass{article}

\usepackage[portuguese]{babel}
\usepackage[utf8]{inputenc}

\title{Relatório BDAD}
\date{31-3-2015}
\author{João Cabral, up20130495\\
	   João Mota, up201303462\\
	   Luís Morais, up2013xxxxx}

\begin{document}

\maketitle
\pagenumbering{gobble}
\newpage
\pagenumbering{arabic}

\section{Esquema Relacional}
Pessoa(\underline{idPessoa}, nome)
Prisioneiro(\underline{número}, idPessoa $\rightarrow$ Pessoa, data de entrada, data de saída, cela $\rightarrow$ Cela)\\
Funcionário(\underline{idFuncionário}, idPessoa $\rightarrow$ Pessoa, cargo, hora de entrada, hora de saída)\\
Tempo de Recreio(\underline{idRecompensa $\rightarrow$ Recompensa}, duração, prisioneiro $\rightarrow$ Prisioneiro)\\
Objeto Pessoal(\underline{idRecompensa $\rightarrow$ Recompensa}, descrição, prisioneiro $\rightarrow$ Prisioneiro)\\
Local(\underline{idLocal},nome, função)\\
Cela(\underline{número}, idLocal $\rightarrow$ local)\\
Pena(\underline{idPena}, data de sentença, motivo, prisioneiro $\rightarrow$ Prisioneiro)\\
Incidente(\underline{idIncidente}, descrição, data, local $\rightarrow$ Local, relator $\rightarrow$ Funcionário )\\
PrisioneiroIncidente(\underline{prisioneiro $\rightarrow$ Prisioneiro}, \underline{incidente $\rightarrow$ Incidente})\\
Penalização(\underline{idPena}$\rightarrow$ Pena, \underline{idIncidente} $\rightarrow$ Incidente, descrição)\\
Recompensa(\underline{idRecompensa}, motivo, idPrisioneiro $\rightarrow$ Prisioneiro)\\

%Cmo se faz pra classes d atribuição?

\section{SQL-LDD}


CREATE TABLE Pessoa (\newline
idPessoa	INTEGER PRIMARY KEY,\newline
nome		NVARCHAR(50)\newline
);\\

\noindent
CREATE TABLE Prisioneiro (\newline
idPessoa	INTEGER PRIMARY KEY REFERENCES Pessoa(idPessoa),\newline
dataDeEntrada	INTEGER,\newline
dataDeSaida	INTEGER,\newline
cela		INETEGER REFERENCES Cela(número),\newline
CHECK	(dataDeEntrada \textless  dataDeSaida)\newline
);\newline

CREATE TABLE Funcionário (\newline
idPessoa	INTEGER REFERENCES Pessoa(idPessoa),\newline
cargo		NVARCHAR(20),\\
horaDeEntrada	INTEGER,\\
horaDeSaída	INTEGER,\\
CHECK (horaDeEntrada \textless horaDeSaida)\\
);
\newpage
CREATE TABLE TempoDeRecreio(\\
idRecompensa	INTEGER PRIMARY KEY REFERENCES Recompensa(idRecompensa),\\
duração		INTEGER,\\
prisioneiro	INTEGER REFERENCES Prisioneiro(idPessoa)\\
);\\

CREATE TABLE ObjetoPessoal(\\
idRecompensa	INTEGER PRIMARY KEY REFERENCES Recompensa(idRecompensa),\\
descrição	NVARCHAR(50),\\
prisioneiro 	INTEGER REFERENCES Prisioneiro(idPessoa)\\
);\\

CREATE TABLE Local(\\
idLocal		INTEGER PRIMARY KEY,\\
nome		NVARCHAR(50),\\
função		NVARCHAR(100)\\
);\\

CREATE TABLE Cela(\\
número		INTEGER PRIMARY KEY,\\
idLocal		INTEGER REFERENCES LOCAL(idLocal)\\
);\\

CREATE TABLE Pena(\\
idPena 		INTEGER PRIMARY KEY,\\
dataDeSentença	INTEGER,\\
motivo		NVARCHAR(100),\\
idPessoa	INTEGER REFERENCES Prisioneiro(idPessoa)\\
);

CREATE TABLE Incidente(\\
idIncidente 	INTEGER PRIMARY KEY,\\
descrição	NVARCHAR(100),\\
data		INTEGER,
idLocal		INTEGER REFERENCES Local(idLocal),\\
relator		INTEGER REFERENCES Funcionário(idPessoa)\\
);

CREATE TABLE PrisioneiroIncidente(\\
prisioneiro	INTEGER REFERENCES Pessoa(idPessoa),\\
incidente	INTEGER REFERENCES Incidente(idIncidente),\\
PRIMARY KEY(prisioneiro, incidente)\\
);\\
\newpage
CREATE TABLE Penalização(\\
idPena 		INTEGER REFERENCES Pena(idPena),\\
idIncidente	INTEGER REFERENCES Incidente(idIncidente),\\
descrição	NVARCHAR(100),\\
PRIMARY KEY (idPena, idIncidente)\\
);\\

CREATE TABLE Recompensa(\\
idRecompensa	INTEGER PRIMARY KEY,
motivo 		NVARCHAR(100),
idPrisioneiro	INTEGER REFERENCES Prisioneiro(idPessoa)
);





	
\end{document}
