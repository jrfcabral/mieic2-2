\documentclass[a4paper,12pt,titlepage]{article}
\usepackage[portuguese]{babel}
\usepackage[utf8]{inputenc}
\usepackage[T1]{fontenc}
\usepackage{mathtools}
\usepackage{amssymb}
\usepackage{amsmath}
\usepackage{url}
\let\biconditional\leftrightarrow
\begin{document}

\title{Relatório preliminar de Conceção e Análise de Algoritmos}
\date{março 2015}
\author{Francisco Veiga, 201201604@fe.up.pt
 \and João Cabral, up201304395@fe.up.pt
 \and  João Mota, 201303462@fe.up.pt\linebreak

 \and Faculdade de Engenharia da Universidade do Porto}
%%%%%%%%%%%%%%%%%%%%%%%%%%%%%%%%%%%%%%%%%%
% University Assignment Title Page 
% LaTeX Template
% Version 1.0 (27/12/12)
%
% This template has been downloaded from:
% http://www.LaTeXTemplates.com
%
% Original author:
% WikiBooks (http://en.wikibooks.org/wiki/LaTeX/Title_Creation)
%
% License:
% CC BY-NC-SA 3.0 (http://creativecommons.org/licenses/by-nc-sa/3.0/)
% 
% Instructions for using this template:
% This title page is capable of being compiled as is. This is not useful for 
% including it in another document. To do this, you have two options: 
%
% 1) Copy/paste everything between \begin{document} and \end{document} 
% starting at \begin{titlepage} and paste this into another LaTeX file where you 
% want your title page.
% OR
% 2) Remove everything outside the \begin{titlepage} and \end{titlepage} and 
% move this file to the same directory as the LaTeX file you wish to add it to. 
% Then add %%%%%%%%%%%%%%%%%%%%%%%%%%%%%%%%%%%%%%%%%
% University Assignment Title Page 
% LaTeX Template
% Version 1.0 (27/12/12)
%
% This template has been downloaded from:
% http://www.LaTeXTemplates.com
%
% Original author:
% WikiBooks (http://en.wikibooks.org/wiki/LaTeX/Title_Creation)
%
% License:
% CC BY-NC-SA 3.0 (http://creativecommons.org/licenses/by-nc-sa/3.0/)
% 
% Instructions for using this template:
% This title page is capable of being compiled as is. This is not useful for 
% including it in another document. To do this, you have two options: 
%
% 1) Copy/paste everything between \begin{document} and \end{document} 
% starting at \begin{titlepage} and paste this into another LaTeX file where you 
% want your title page.
% OR
% 2) Remove everything outside the \begin{titlepage} and \end{titlepage} and 
% move this file to the same directory as the LaTeX file you wish to add it to. 
% Then add %%%%%%%%%%%%%%%%%%%%%%%%%%%%%%%%%%%%%%%%%
% University Assignment Title Page 
% LaTeX Template
% Version 1.0 (27/12/12)
%
% This template has been downloaded from:
% http://www.LaTeXTemplates.com
%
% Original author:
% WikiBooks (http://en.wikibooks.org/wiki/LaTeX/Title_Creation)
%
% License:
% CC BY-NC-SA 3.0 (http://creativecommons.org/licenses/by-nc-sa/3.0/)
% 
% Instructions for using this template:
% This title page is capable of being compiled as is. This is not useful for 
% including it in another document. To do this, you have two options: 
%
% 1) Copy/paste everything between \begin{document} and \end{document} 
% starting at \begin{titlepage} and paste this into another LaTeX file where you 
% want your title page.
% OR
% 2) Remove everything outside the \begin{titlepage} and \end{titlepage} and 
% move this file to the same directory as the LaTeX file you wish to add it to. 
% Then add \input{./title_page_1.tex} to your LaTeX file where you want your
% title page.
%
%%%%%%%%%%%%%%%%%%%%%%%%%%%%%%%%%%%%%%%%%

%----------------------------------------------------------------------------------------
%	PACKAGES AND OTHER DOCUMENT CONFIGURATIONS
%----------------------------------------------------------------------------------------

\begin{titlepage}

\newcommand{\HRule}{\rule{\linewidth}{0.5mm}} % Defines a new command for the horizontal lines, change thickness here

\center % Center everything on the page
 
%----------------------------------------------------------------------------------------
%	HEADING SECTIONS
%----------------------------------------------------------------------------------------

\textsc{\LARGE Faculdade de Engenharia da Universidade do Porto}\\[1.5cm] % Name of your university/college
\textsc{\Large Mestrado Integrado em Engenharia Informática e da Computação}\\[0.5cm] % Major heading such as course name
\textsc{\large Conceção e Análise de Algoritmos}\\[0.5cm] % Minor heading such as course title

%----------------------------------------------------------------------------------------
%	TITLE SECTION
%----------------------------------------------------------------------------------------

\HRule \\[0.4cm]
{ \huge \bfseries Trabalho Prático - Relatório Preliminar}\\[0.4cm] % Title of your document
\HRule \\[1.5cm]
 
%----------------------------------------------------------------------------------------
%	AUTHOR SECTION
%----------------------------------------------------------------------------------------
% If you don't want a supervisor, uncomment the two lines below and remove the section above
\Large \emph{Author:}\\
Francisco Veiga, 201201604@fe.up.pt
 \and João Cabral, up201304395@fe.up.pt
 \and  João Mota, 201303462@fe.up.pt\linebreak\\[3cm] % Your name


%----------------------------------------------------------------------------------------
%	DATE SECTION
%----------------------------------------------------------------------------------------

{\large \today}\\[3cm] % Date, change the \today to a set date if you want to be precise

%----------------------------------------------------------------------------------------
%	LOGO SECTION
%----------------------------------------------------------------------------------------

%\includegraphics{Logo}\\[1cm] % Include a department/university logo - this will require the graphicx package
 
%----------------------------------------------------------------------------------------

\vfill % Fill the rest of the page with whitespace

\end{titlepage}

 to your LaTeX file where you want your
% title page.
%
%%%%%%%%%%%%%%%%%%%%%%%%%%%%%%%%%%%%%%%%%

%----------------------------------------------------------------------------------------
%	PACKAGES AND OTHER DOCUMENT CONFIGURATIONS
%----------------------------------------------------------------------------------------

\begin{titlepage}

\newcommand{\HRule}{\rule{\linewidth}{0.5mm}} % Defines a new command for the horizontal lines, change thickness here

\center % Center everything on the page
 
%----------------------------------------------------------------------------------------
%	HEADING SECTIONS
%----------------------------------------------------------------------------------------

\textsc{\LARGE Faculdade de Engenharia da Universidade do Porto}\\[1.5cm] % Name of your university/college
\textsc{\Large Mestrado Integrado em Engenharia Informática e da Computação}\\[0.5cm] % Major heading such as course name
\textsc{\large Conceção e Análise de Algoritmos}\\[0.5cm] % Minor heading such as course title

%----------------------------------------------------------------------------------------
%	TITLE SECTION
%----------------------------------------------------------------------------------------

\HRule \\[0.4cm]
{ \huge \bfseries Trabalho Prático - Relatório Preliminar}\\[0.4cm] % Title of your document
\HRule \\[1.5cm]
 
%----------------------------------------------------------------------------------------
%	AUTHOR SECTION
%----------------------------------------------------------------------------------------
% If you don't want a supervisor, uncomment the two lines below and remove the section above
\Large \emph{Author:}\\
Francisco Veiga, 201201604@fe.up.pt
 \and João Cabral, up201304395@fe.up.pt
 \and  João Mota, 201303462@fe.up.pt\linebreak\\[3cm] % Your name


%----------------------------------------------------------------------------------------
%	DATE SECTION
%----------------------------------------------------------------------------------------

{\large \today}\\[3cm] % Date, change the \today to a set date if you want to be precise

%----------------------------------------------------------------------------------------
%	LOGO SECTION
%----------------------------------------------------------------------------------------

%\includegraphics{Logo}\\[1cm] % Include a department/university logo - this will require the graphicx package
 
%----------------------------------------------------------------------------------------

\vfill % Fill the rest of the page with whitespace

\end{titlepage}

 to your LaTeX file where you want your
% title page.
%
%%%%%%%%%%%%%%%%%%%%%%%%%%%%%%%%%%%%%%%%%

%----------------------------------------------------------------------------------------
%	PACKAGES AND OTHER DOCUMENT CONFIGURATIONS
%----------------------------------------------------------------------------------------

\begin{titlepage}

\newcommand{\HRule}{\rule{\linewidth}{0.5mm}} % Defines a new command for the horizontal lines, change thickness here

\center % Center everything on the page
 
%----------------------------------------------------------------------------------------
%	HEADING SECTIONS
%----------------------------------------------------------------------------------------

\textsc{\LARGE Faculdade de Engenharia da Universidade do Porto}\\[1.5cm] % Name of your university/college
\textsc{\Large Mestrado Integrado em Engenharia Informática e da Computação}\\[0.5cm] % Major heading such as course name
\textsc{\large Conceção e Análise de Algoritmos}\\[0.5cm] % Minor heading such as course title

%----------------------------------------------------------------------------------------
%	TITLE SECTION
%----------------------------------------------------------------------------------------

\HRule \\[0.4cm]
{ \huge \bfseries Trabalho Prático - Relatório Preliminar}\\[0.4cm] % Title of your document
\HRule \\[1.5cm]
 
%----------------------------------------------------------------------------------------
%	AUTHOR SECTION
%----------------------------------------------------------------------------------------
% If you don't want a supervisor, uncomment the two lines below and remove the section above
\Large \emph{Author:}\\
Francisco Veiga, 201201604@fe.up.pt
 \and João Cabral, up201304395@fe.up.pt
 \and  João Mota, 201303462@fe.up.pt\linebreak\\[3cm] % Your name


%----------------------------------------------------------------------------------------
%	DATE SECTION
%----------------------------------------------------------------------------------------

{\large \today}\\[3cm] % Date, change the \today to a set date if you want to be precise

%----------------------------------------------------------------------------------------
%	LOGO SECTION
%----------------------------------------------------------------------------------------

%\includegraphics{Logo}\\[1cm] % Include a department/university logo - this will require the graphicx package
 
%----------------------------------------------------------------------------------------

\vfill % Fill the rest of the page with whitespace

\end{titlepage}


\maketitle
\tableofcontents
\newpage
\section{Introdução}

No contexto da unidade curricular de Conceção e Análise de Algoritmos foi solicitada a resolução de um problema relacionado com a distribuição de uma rede de fibra ótica pela rede habitacional de um determinado agregado populacional.

\section{Problemas a abordar} 
Numa primeira fase é solicitada uma aplicação que receba como dados de entrada um mapa do referido agregado populacional e produza como saída uma representação gráfica, sob a forma de um gráfico, de uma distribuição ideal da rede de fibra ótica, minimizando o comprimento das ligações utilizadas. Estabelece-se ainda como restrição adicional que cada uma das casas cobertas não pode situar-se fora de uma determinada área a ser definida por uma distância máxima à central de onde parte a rede de fibra ótica.
 
Numa segunda fase é solicitado que a aplicação alargue o raio de ação de cobertura da rede de fibra ótica, procurando abranger uma maior área, sendo contudo necessário que a aplicação seja capaz de detetar áreas onde a cobertura providenciada por uma única central se revele insuficiente e seja capaz de indicar a necessidade de existirem novas centrais de distribuição da rede.

\newpage
\section{Formalização do problema}

\subsection{1º problema/fase}
\subsubsection*{Inputs}
\begin{itemize}
\item Um grafo  $G= ( V, E ) $ conexo, onde $V$ é o conjunto das casas e $E$ o conjunto das suas ligações e para cada aresta $(u,v)\in E$ temos $w(u,v)$ representando a distância entre as arestas $u$ e $v$ e $d(p)$ como sendo a distância de um caminho $p = \langle v_0,v_1,\ldots , v_k\rangle$\cite[p.~624]{intro_algo};
\item Uma distancia $l:l > 0$;
\item Um vértice $s:s\in V$.
\end{itemize}

\subsubsection*{Outputs}
\begin{itemize}
\item Um grafo $G_T = ( V_T,E_T )$ tal que se verifique a condição\cite[p.~643]{intro_algo}:
$$ \forall v \in V(\delta(s,v) \leq l \biconditional v \in V_T)$$
$$\delta(s,v) = 
\begin{cases}
\min \{d(p): s \leadsto^p v\} & \text{se existe caminho entre s e v}\\
\infty & \text{caso contrário} 
\end{cases}$$
\end{itemize}

\subsubsection*{Função Objetivo}
Seja $$x(u,v) = \begin{cases}	
1 & \text{se a aresta} (u,v) \in E_T\\
0 & \text{caso contrário} 
\end{cases}$$

A função objetivo é\cite{ieor_mst}
$$\min \sum_{(u,v)\in E} w(u,v)x(u,v)$$

com as restrições 
$$\sum_{(u,v)\in E} x(u,v) = |V_T| - 1$$
$$\sum_{(u,v)\in (S,S)} x(u,v) \leq |S| - 1 \quad \forall S \subseteq  V_T$$

\subsection{2º problema/fase}
\subsubsection*{Inputs}
\begin{itemize}
\item Um grafo  $G= ( V, E ) $, desconexo ou não, onde $V$ é o conjunto das casas e $E$ o conjunto das suas ligações e para cada aresta $(u,v)\in E$ temos $w(u,v)$ representando a distância entre as arestas $u$ e $v$ e $d(p)$ como sendo a distância de um caminho $p = \langle v_0,v_1,\ldots , v_k\rangle$\cite[p.~624]{intro_algo};
\end{itemize}

\subsubsection*{Outputs}
\begin{itemize}
\item Um grafo $G_T = ( V_T,E_T )$;
\item Um conjunto de grafos $D \subseteq G_T = \{d(V_D,E_D): \forall d,e \in D(\nexists v \in V_T(v \in d \land v \in e) \lor e=v) \land \nexists d,e\in D(\exists(u,v) \in E_T(u \in d \land v \in e)) \land u\neq v \}$
\item Um conjunto $S = \{s : \exists! d \in D(s \in d)\}$.
\end{itemize}

\subsubsection*{Função Objetivo}
Seja $$x(u,v) = \begin{cases}
1 & \text{se a aresta} (u,v) \in E_T\\
0 & \text{caso contrário} 
\end{cases}$$

A função objetivo é\cite{ieor_mst}
$$\min \sum_{(u,v)\in E} w(u,v)x(u,v)$$

com as restrições 
$$\sum_{(u,v)\in E} x(u,v) = |V_T| - 1$$
$$\sum_{(u,v)\in (S,S)} x(u,v) \leq |S| - 1 \quad \forall S \subseteq V_T$$
\linebreak
\linebreak
\linebreak
\linebreak
\section{Ideias de solução}
\subsection{1º problema/fase}
Aplicar o \emph{algoritmo de Dijkstra}, um algoritmo ganancioso, para calcular o caminho mais curto para todos os pontos do grafo a partir da central de forma a aferir os vértices do grafo que se encontram dentro do raio de cobertura pretendido. Excluir do grafo todos os pontos cuja distância à central seja superior ao raio desejado.

De seguida usar o \emph{algoritmo de Prim}, também ele um algoritmo ganancioso, para calcular uma árvore de expansão mínima do grafo resultante da operação anterior, resultando numa distância ótima de cabo a utilizar.

\subsection{2º problema/fase}
Para a segunda fase prevê-se um aumento do raio de abrangência para a central. Isto pode levar à existência de sub-grafos desconexos. É da nossa interpretação que a cada grafo desconexo se deve acrescentar uma central. O \emph{algoritmo de Prim} só permite chegar à árvore de expansão mínima de um dos sub-grafos desconexos de $G$ por cada vez que é corrido. Embora existam algoritmos que permitem calcular árvores de expansão mínima em grafos desconexos, gerando uma árvore em cada sub-grafo, o cálculo do conjunto $D$ essencial para obter $S$ implica necessariamente uma determinação destes subgrafos.




\section{Métricas de avaliação}
Para obter uma métrica da eficiência espacial e temporal dos algoritmos pretende-se aplicar chamadas às funções do sistema operativo que permitem obter informações acerca do tempo de execução e da ocupação de memória de um determinado código. Se no caso da eficiência temporal esta informação é de aplicação mais direta, podendo ser obtida imediatamente antes e depois da execução do algoritmo cuja eficiência se pretende medir, no caso da eficiência espacial torna-se essencial escolher pontos relevantes para efetuar a medição, dado que ao longo da execução de um programa a sua ocupação em memória pode variar drasticamente.
 
É ainda importante ter em conta para a medição do tempo de execução de um processo que as chamadas ao sistema que permitem obter a sua ocupação em memória têm um tempo de execução não nulo, pelo que convém efetuar as duas medições separadamente.
\subsection{Variação das entradas}
\subsubsection{1º problema/fase}
Torna-se importante variar o número de nós do grafo bem como o seu número de arestas separadamente, para aferir do impacto de cada uma das variáveis no tempo de execução de cada um dos algoritmos aplicados. Torna-se também importante variar o número de arestas a serem excluídas pela aplicação do algoritmo de Dijkstra antes do cálculo da árvore de expansão mínima. 
\subsubsection{2º problema/fase}
Aplica-se a mesma necessidade de variar o número de grafos e o número de nós que se verifica na 1ª fase. Torna-se também importante variar a conectividade dos grafos, testando os algoritmos com um número variável de subgrafos desconexos, para avaliar o impacto que isso tem na performance de cada um.




\newpage
\bibliographystyle{plain}
\bibliography{cal}


\end{document}
