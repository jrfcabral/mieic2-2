\documentclass[a4paper,12pt,titlepage]{article}
\usepackage[portuguese]{babel}
\usepackage[utf8]{inputenc}
\usepackage[T1]{fontenc}
\usepackage{mathtools}
\usepackage{amssymb}
\usepackage{amsmath}

\let\biconditional\leftrightarrow
\begin{document}
\title{Relatório preliminar de Conceção e Análise de Algoritmos}
\date{março 2015}
\author{Francisco Veiga, up201304395@fe.up.pt
 \and João Mota, up201304395@fe.up.pt
 \and João Cabral, up201304395@fe.up.pt}
\maketitle
\tableofcontents
\newpage
\section{Introdução}

No contexto da unidade curricular de Conceção e Análise de Algoritmos foi solicitada a resolução de um problema relacionado com a distribuição de uma rede de fibra ótica pela rede habitacional de um determinado agregado populacional.

\section{Problemas a abordar} 
Numa primeira fase é solicitada uma aplicação que receba como dados de entrada um mapa do referido agregado populacional e produza como saída uma representação gráfica, sob a forma de um gráfico, de uma distribuição ideal da rede de fibra ótica, minimizando o comprimento das ligações utilizadas. Estabelece-se ainda como restrição adicional que cada uma das casas cobertas não pode situar-se fora de uma determinada área a ser definida por uma distância máxima à central de onde parte a rede de fibra ótica.
 
Numa segunda fase é solicitado que a aplicação alargue o raio de ação de cobertura da rede de fibra ótica, procurando abranger uma maior área, sendo contudo necessário que a aplicação seja capaz de detetar áreas onde a cobertura providenciada por uma única central se revele insuficiente e seja capaz de indicar a necessidade de existirem novas centrais de distribuição da rede.

\newpage
\section{Formalização do problema}

\subsection{1º problema/fase}
\subsubsection*{Inputs}
\begin{itemize}
\item Um grafo  $G= ( V, E ) $ conexo, onde $V$ é o conjunto das casas e $E$ o conjunto das suas ligaçoes e para cada aresta $(u,v)\in E$ temos $w(u,v)$ representando a distância entre as arestas $u$ e $v$ e $d(p)$ como sendo a distância de um caminho $p = \langle v_0,v_1,\ldots , v_k\rangle$;
\item Uma distancia $l:l > 0$;
\item Um vértice $s:s\in V$.
\end{itemize}

\subsection*{Outputs}
\begin{itemize}
\item Um grafo $G_T = ( V_T,E_T )$ tal que se verifique a condição
$$ \forall v \in V(\delta(s,v) \leq l \biconditional v \in V_T)$$
$$\delta(s,v) = 
\begin{cases}
\min \{d(p): s \leadsto^p v\} & \text{se existe caminho entre s e v}\\
\infty & \text{caso contrário} 
\end{cases}$$
\end{itemize}

\subsection*{Função Objetivo}
Seja $$x(u,v) = \begin{cases}
1 & \text{se a aresta} (u,v) \in E_T\\
0 & \text{caso contrário} 
\end{cases}$$

A função objetivo é
$$\min \sum_{(u,v)\in E} w(u,v)x(u,v)$$

com as restrições 
$$\sum_{(u,v)\in E} x(u,v) = |V_T| - 1$$
$$\sum_{(u,v)\in (S,S)} x(u,v) \leq |S| - 1 \quad \forall S \subseteq  V_T$$

\subsection{2º problema/fase}
\subsubsection*{Inputs}
\begin{itemize}
\item Um grafo  $G= ( V, E ) $ desconexo, onde $V$ é o conjunto das casas e $E$ o conjunto das suas ligaçoes e para cada aresta $(u,v)\in E$ temos $w(u,v)$ representando a distância entre as arestas $u$ e $v$ e $d(p)$ como sendo a distância de um caminho $p = \langle v_0,v_1,\ldots , v_k\rangle$;
\end{itemize}

\subsection*{Outputs}
\begin{itemize}
\item Um grafo $G_T = ( V_T,E_T )$;
\item Um conjunto de grafos $D \subseteq G_T = \{d(V_D,E_D): \forall d,e \in D(\nexists v \in V_T(v \in d \land v \in e)) \land \nexists d,e\in D(\exists(u,v) \in E_T(u \in d \land v \in e)) \}$
\item Um conjunto $S = \{s : \exists! d \in D(s \in d)\}$.
\end{itemize}

\subsection*{Função Objetivo}
Seja $$x(u,v) = \begin{cases}
1 & \text{se a aresta} (u,v) \in E_T\\
0 & \text{caso contrário} 
\end{cases}$$

A função objetivo é
$$\min \sum_{(u,v)\in E} w(u,v)x(u,v)$$

com as restrições 
$$\sum_{(u,v)\in E} x(u,v) = |V_T| - 1$$
$$\sum_{(u,v)\in (S,S)} x(u,v) \leq |S| - 1 \quad \forall S \subseteq V_T$$




\end{document}
